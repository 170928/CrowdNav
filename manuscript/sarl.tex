%%%%%%%%%%%%%%%%%%%%%%%%%%%%%%%%%%%%%%%%%%%%%%%%%%%%%%%%%%%%%%%%%%%%%%%%%%%%%%%%
%2345678901234567890123456789012345678901234567890123456789012345678901234567890
%        1         2         3         4         5         6         7         8

\documentclass[letterpaper, 10 pt, conference]{ieeeconf}  % Comment this line out if you need a4paper

%\documentclass[a4paper, 10pt, conference]{ieeeconf}      % Use this line for a4 paper

\IEEEoverridecommandlockouts                              % This command is only needed if 
                                                          % you want to use the \thanks command

\overrideIEEEmargins                                      % Needed to meet printer requirements.

% See the \addtolength command later in the file to balance the column lengths
% on the last page of the document

% The following packages can be found on http:\\www.ctan.org
\usepackage{graphics} % for pdf, bitmapped graphics files
\usepackage{epsfig} % for postscript graphics files
\usepackage{mathptmx} % assumes new font selection scheme installed
\usepackage{times} % assumes new font selection scheme installed
\usepackage{amsmath} % assumes amsmath package installed
\usepackage{amssymb}  % assumes amsmath package installed
% \usepackage{comment}
% \usepackage{algorithm,algpseudocode}
% \usepackage{resizegather}
% \usepackage{flexisym}
\usepackage{dblfloatfix}    % To enable figures at the bottom of pag
\usepackage{subcaption}


\title{\Large \bf
Social Navigation}

\author{% <-this % stops a space
\thanks{*This work was supported by [?]}% <-this % stops a space
% \thanks{$^{1}$L is with Robotics group, Segway Inc., 
%         DS Tech A503, # Postcode, Beijing, China
%         {\tt\small albert.author@papercept.net}}%
% \thanks{$^{2}$X with the Department of Electrical Engineering, Tsinghua University, 
%         Address, # Postcode, China
% {\tt\small b.d.researcher@ieee.org}}%
% \thanks{$^{*}$Equal contribution}       
\thanks{Visual Intelligence for Transportation Laboratory, Ecole Polytechnique Federale de Lausanne (EPFL)
, CH-1015 Lausanne,
        {\tt\small \{ \}@epfl.ch}}%
}

\begin{document}

\bstctlcite{IEEEexample:BSTcontrol}

\maketitle
\thispagestyle{empty}
\pagestyle{empty}

\pdfminorversion=4  
%%%%%%%%%%%%%%%%%%%%%%%%%%%%%%%%%%%%%%%%%%%%%%%%%%%%%%%%%%%%%%%%%%%%%%%%%%%%%%%%
\begin{abstract}
This electronic document is a template. The various components of your paper [title, text, heads, etc.] are already defined on the style sheet, as illustrated by the portions given in this document.

\end{abstract}


%%%%%%%%%%%%%%%%%%%%%%%%%%%%%%%%%%%%%%%%%%%%%%%%%%%%%%%%%%%%%%%%%%%%%%%%%%%%%%%%
\section{INTRODUCTION} \label{sec:intro}

\vspace{10cm}

\section{BACKGROUND} \label{sec:background} 

RL, 

Social interaction, 

\section{APPROACH} \label{sec:approach} 

Formulation, 

\section{RESULTS} \label{sec:results} 

Experiments, 

\section{CONCLUSION} \label{sec:conclusion} 

In this paper, 

% \subsection{System Structure}

\addtolength{\textheight}{-10cm}   % This command serves to balance the column lengths
                                  % on the last page of the document manually. It shortens
                                  % the textheight of the last page by a suitable amount.
                                  % This command does not take effect until the next page
                                  % so it should come on the page before the last. Make
                                  % sure that you do not shorten the textheight too much.

\section*{ACKNOWLEDGMENT}

We would like to thank ... 

\bibliographystyle{IEEEtran}
% \bibliography{Planning}

% \begin{thebibliography}{99}

% \end{thebibliography}

\end{document}
